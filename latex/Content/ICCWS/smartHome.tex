The smart home architecture includes three controller components and various connected devices. The controller components include (i) a Raspberry Pi running the Homebridge server that emulates the iOS HomeKit API and exposes supported devices to Apple's HomeKit, (ii) an iPhone 6+ running Apple's HomeKit and device specific applications, and (iii) an Apple TV Generation 2 acting as a smart home hub to allow access to HomeKit supported devices while the user is away from the smart home. The communication between controllers and devices can be observed in Figure~\ref{fig:SmartHomeArchitecture} and is described in the rest of this section.

\figSmartHomeArchitecture

\subsubsection{Raspberry Pi}

The Raspberry Pi 3 Model B with Raspbian Jessie Lite version 4.9 operating system is connected to the smart home network via the on-board 802.11 b/g/n 2.4 GHz wireless chip \cite{rasbperry}. The Raspberry Pi runs Homebridge version 0.4.14 as a systemd service and each interaction between a controller and device is logged in the systemd journal \cite{homebridge}. A plugin is utilized to expose WeMo devices to the Apple Homekit and is loaded into Homebridge \cite{wemo}. The Apple devices communicate with the Raspberry Pi to interact with the WiFi Devices w$ _2 $-w$ _9 $.

\subsubsection{Apple Devices}

The iPhone 6+ and Apple TV act as controllers in the smart home architecture and connect to devices via WiFi and \ac{BLE}. When the user is home, the iPhone connects to WiFi devices via the Homebridge and connects directly to the \ac{BLE} devices. Some of the \ac{BLE} devices are not supported by Apple's Homekit and can only be accessed through the manufacturer provided iOS application on the iPhone. When the user is away from the smart home, the iPhone can communicate with Homekit supported devices via the iCloud and Apple TV acting as a hub. For example, if the user is away from home and wants to access the temperature in a room, the iPhone communicates with the Apple TV via the iCloud and the Apple TV will communicate with the device in the home via WiFi or \ac{BLE}. This will only work with Homekit supported devices, therefore, \ac{BLE} devices b$ _7 $-b$ _{12} $ (see \ref{tbl:BtleDevices}) cannot be accessed while the user is away from the home. Devices will be accessed while the user is both home and away to observe differences in communication.

\subsubsection{WiFi Devices}

To facilitate WiFi communication in the smart home architecture, a 2.4GHz WiFi \ac{AP}, ``Prancing Pony", was setup with WPA2 security on channel 1 (see Figure~\ref{fig:AccessPoint} for a complete list of settings). A list of devices connected to the \ac{AP} can be found in Table~\ref{tbl:WifiDevices}. The smart home devices include a camera, six outlets (four smart plugs, one mini plug, and one energy plug), and a motion sensor (w$ _2 $-w$ _9 $). These devices use the Homebridge to communicate with Apple's HomeKit on the iPhone. 

\figAccessPoint

\tableWifiDevices

\subsubsection{Bluetooth Low Energy Devices}

For Bluetooth communication to occur in the smart home architecture a Bluetooth master must be present. In the smart home architecture, the iPhone and Apple TV act as masters while each of the \ac{BLE} devices are slaves. A list of devices operating in the \ac{BLE} can be found in Table~\ref{tbl:BtleDevices}. The \ac{MAC} addresses for each device are not included because they are randomized per the \ac{BLE} protocol \cite{sig4.2}.

\tableBtleDevices