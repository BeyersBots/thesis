\subsection{Applications}

\subsection{Recommendations}
Periodically changing MAC addresses
Encrypting lower-layer data packets
Chaffing and Winnowing

\subsection{Vulnerability Drivers}

While the recommendations in the previous section can improve the security of smart home environments, none of these ideas are new. Why, then, have these fixes not been implemented to secure privacy? For example, the \ac{BLE} specification defines security procedures to encrypt the payload, generate private addresses, and provide authentication \cite{sig4.2}. However, implementation of security is left up to the designer and each additional security measure contributes to increased energy consumption \cite{rHeydon}. Limiting power, developing devices quickly, and other design constraints drive developers towards poor security implementation, leaving devices with essentially no Link Layer authentication or encryption. Also, while network and computer security has seen more adversarial pressure, the smart home is relatively new. Until recently outlets, switches, and light-bulbs have not been connected to networks. This is the same evolution vehicles are facing as they become more connected to the Internet.

\subsection{Limitations}