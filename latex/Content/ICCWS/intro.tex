In recent years, smart home devices have become one of the most popular categories for the \ac{IoT} accounting for \$3.5 billion of a \$292 billion industry; over 29 million smart home devices are expected to ship in 2017, a 63 percent increase over 2016 \cite{consumerTech}. WiFi and \ac{BLE} are two of the primary protocols used in these devices and are commonly implemented in security cameras, locks, medical devices, sensors, and a myriad of other devices. With the increasing prevalence of \ac{BLE} and WiFi devices in the home, consumers must be aware of the information these devices inadvertently broadcast and what kind of privacy data an outside observer can infer. 

This work contributes to the field of \ac{IoT} security, specifically privacy within a smart home, by illustrating how an observer can use information leaked to classify devices, identify events such as when a door is opened or when a light is turned on, and track occupants of the home. In doing so, we make four principal contributions:

\textbf{Smart home architecture}. To analyze \ac{IoT} data leakage in the wild, we provide a realistic smart home architecture that integrates WiFi and \ac{BLE} \ac{COTS} devices with Apple's home automation application, HomeKit. Examples of interactions with the smart home environment include turning on lights, opening doors, activating motion sensors, and unlocking locks. These interactions occur both in a controlled manner and freely and while the user is home or away.

\textbf{Vulnerability analysis.} We explain how an eavesdropper can use device vulnerabilities to extract information while outside the smart home environment via raw signals sniffed over the air.  These, combined with characteristic data exchanges and packet sizes, can be used to fingerprint components of the smart home environment.

\textbf{CITIoT.} The tool presented in this work provides three capabilities against smart home environments: device classification, event identification, and user tracking. For WiFi, a fingerprinting technique is applied to classify smart home devices into one of three groups: sensor, outlet, or camera. For \ac{BLE} devices we similarly classify devices, but provide more descriptive information. The fingerprint technique is also applied to identify events within the smart home such as turning on a light or movement in the house. Lastly, the observed smart home traffic is used to predict when users will be in the smart home.

\textbf{Synthesis.} We present how an observer can use the information gathered from smart home devices to create pattern-of-life models, crack a Bluetooth lock, and gain access to the home when a user is predicted to be away. We observe that these vulnerabilities are not unique to the devices under study, but indicative of \ac{IoT} design. We provide limitations to our approach and recommendations to prevent these vulnerabilities and create a more secure smart home environment.

The remainder of this paper is organized as follows: Section~\ref{background} discusses smart home technologies, a brief technical overview of the WiFi and \ac{BLE} protocols, and a summary of tools leveraged in the design of CITIoT. Section~\ref{relatedWork} summarizes related research in the areas of WiFi and \ac{BLE} security and \ac{IoT} device privacy. Section~\ref{smartHome} presents the smart home architecture used in creating and analyzing CITIoT. Section~\ref{methodology} describes the process used to create CITIoT starting with methods of reconnaissance and device traffic sniffing, to detailing the construction of the fingerprinting method used to classify, identify, and track devices. Section~\ref{results} reports CITIoT's accuracy on our smart home architecture. Finally, applications, recommendations, vulnerability drivers, and limitations are discussed in Section~\ref{synthesis} before the paper is concluded in Section~\ref{conclusion}.  