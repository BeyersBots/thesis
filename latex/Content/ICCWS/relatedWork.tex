Although Wifi and \ac{BLE} smart home devices are becoming commonplace, the private data leaked and security vulnerabilities of these devices is largely unexplored. In a few recent studies focused on \ac{BLE} wearable fitness trackers one group of researchers observed leaked privacy data to determine user activity level and gait \cite{Das}, while another group tracked a user wearing a Fitbit Surge up to 1,000 meters away with greater than eighty percent accuracy \cite{RoseTrack}. Researchers have used Wifi \ac{MAC} addresses sent in the clear and \ac{RSSI} values to create location tracking systems on campuses, crowd tracking at mass events, and in \ac{CRM} allowing commercial businesses to track customer interactions and data \cite{Zhou}\cite{Bonne}\cite{Atkinson}. The lack of Link Layer authentication or encryption for some smart home \ac{BLE} devices enabled researchers to crack twelve \ac{BLE} locks from up to a quarter mile away \cite{RoseLocks}. Closest to this work, researchers from the United Kingdom were able to use raw Wifi signals to create fingerprinting techniques able to identify applications used on a mobile phone \cite{Atkinson}. From the the extent of our research, however, this is the first work providing a broad review of privacy leakage from \ac{COTS} smart home devices in the wild and using this data to classify devices, identify events, and track users in a smart home.